\documentclass[a4paper,12pt]{article}
\usepackage{graphicx}
\usepackage{titlesec}
\usepackage{biblatex}
\addbibresource{similaritybasedcs.bib}

% Adjust page margins
\usepackage[margin=1in]{geometry}
\linespread{1.5}

% Title formatting
\titleformat{\section}[block]{\normalfont\Large\bfseries}{\thesection}{1em}{}

\begin{document}
\begin{titlepage}
    \centering
    {\huge\bfseries Similarity Based Constraint Score\par}
    \vspace{1cm}
    {\Large Mid-term Research Project Report\par}
    \vspace{1cm}
    {\large Gadegbeku Fabio\par}
    \vspace{1cm}
    \today\par
    \vspace{1cm}
    \raisebox{-0.5\height}{\includegraphics[width=0.2\textwidth]{centralelillelogo.png}}\hspace{0.2cm}
    \raisebox{-0.5\height}{\includegraphics[width=0.2\textwidth]{univlillelogo.png}}\hspace{0.2cm}
    \raisebox{-0.5\height}{\includegraphics[width=0.2\textwidth]{cristallogo.jpg}}\par
\end{titlepage}
\tableofcontents
\newpage
\section{Introduction}
In Machine Learning having too many features is counter productive, this is called the curse of dimensionality.
To avoid this phenomenon there exists feature selection methods that evaluate the revelance of the features. More precisely
in classification problems we can use \textit{must link}\footnote{When two samples have the same class} and
\textit{cannot link}\footnote{When two samples have different classes}
constraints to define constraint scores to evaluate how well each feature
respects the constraints. These constraint scores typically compute distances between the samples in the original
feature space to evaluate them, so still suffer from the curse of dimensionality.
In this report we will present and implement the Similarity Based Constraint Score (SBCS) described by \cite{salmiSimilaritybasedConstraintScore2020}.
That has the unique capibilities of evaluating a whole subset of features at once and calculating distances
in a lower dimensional space.
\subsection*{Goals}
The goals of this project are to first of all implement the SBCS and compare it to other constraint scores on different datasets on
multiple criteria. Secondly we will try to improve the SBCS by using constraints directly instead of available labels to generate
the constraints.
% Your content here...

\section{Laplacian Score}
% Your content here...

\section{Constraint Score 1}
% Your content here...

\section{Constraint Score 4}
% Your content here...
\section{Similarity Based Constraint Score}
% Your content here...
\section{Next Steps}
\printbibliography
\end{document}
